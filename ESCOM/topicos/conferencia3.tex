\documentclass[a4paper,10pt]{article}


%opening
\title{Complejidad y el problema de la migracion}
\author{Allan Ulises Zepeda Ibarra}

\begin{document}

\maketitle

El Dr. Arturo Erdely Ru\'{\i}z aborda el tema de la migraci\'on con relaci\'on al nivel de bienestar humano, el doctor comenta que el nivel de bienestar humano 
depende de varios factores como son el nivel econ\'omico, el nivel de estudios y el nivel de felicidad. Entonces con estas dos variables 
pudi\'eramos pensar primeramente que a un mayor nivel de bienestar humano menor ser\'a la migraci\'on, pero el estudio revela que no es as\'{\i} 
que se comporta de una forma un tanto raro, al momento de analizar los datos se pudieron agrupar en 7 secciones de comportamiento 
diferentes, las cuales se ajustaron o se trataron de explicar con algunas distribuciones de probabilidad. Los resultados al momento de 
explicar el comportamiento describen que en el primer sector la tendencia a migrar no se da porque no tienen los recursos suficientes 
para irse, pero en cuanto su nivel de vida se mejora estos optan por migrar para conseguir un nivel de vida mejor. Pero cuando el nivel 
de vida es m\'as alto que eso se da el comportamiento esperado de que a mayor bienestar humano menor es la tendencia a migrar.
\end{document}
