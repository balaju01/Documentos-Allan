\documentclass[a4paper,10pt]{article}


%opening
\title{Harold V. McIntosh}
\author{Zepeda Ibarra Allan Ulises}

\begin{document}

\maketitle

Obtuvo la Licenciatura en Ciencias con especialidad en F\'{\i}sica de Colorado A\&M College en 1949, la Maestr\'{\i}a en Ciencias 
(en Matem\'aticas) de la Universidad de Cornell en 1952, y termin\'o cr\'editos doctorales en Cornell y Brandeis; obtuvo el Doctorado 
de Filosof\'{\i}a en Qu\'{\i}mica Cu\'antica en la Universidad de Uppsala en 1972.
En M\'exico trabaj\'o en el Departamento de F\'{\i}sica del Centro de Investigaci\'on y Estudios Avanzados del IPN de 1964 a 1965; 
en este per\'{\i}odo dirigió las tesis de Licenciatura de Adolfo Guzm\'an Arenas y Raymundo Segovia Navarro, ambas sobre compiladores para el 
lenguaje de programaci\'on CONVERT, ideado por McIntosh para realizar manipulaciones simb\'olicas \'utiles en la soluci\'on de problemas de 
mec\'anica cl\'asica y cu\'antica.
Entre 1965 y 1966 McIntosh fué director del Departamento de Programación del Centro de Cálculo Electrónico de la UNAM.
De 1966 a 1975 fué Profesor en la Escuela Superior de Física y Matemáticas del IPN. Aquí fué Coordinador de la Academia de Matemáticas 
Aplicadas; a los cursos ya existentes de Análisis Numérico y Probabilidad y Estadística agregó los de Lógica Matemática y Programación.
Con su asesoría se construyeron compiladores de REC para la IBM 1130 y la CDC 3150 en el Centro Nacional de Cálculo del IPN; él mismo se 
hizo cargo de la construcción de paquetes para cálculos matriciales, integración numérica de ecuaciones diferenciales de segundo orden, 
y cálculo de trayectorias de una partícula cargada en el campo de dos centros con cargas magnéticas y eléctricas.

\end{document}

