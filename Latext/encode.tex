\documentclass[a4paper,10pt]{article}
\usepackage[utf8]{inputenc}
\usepackage{graphicx}
\usepackage{listings}
\usepackage{amsmath}
\usepackage{amssymb}

%opening
\title{Mapeo de cadena de caracteres a numero entero}
\author{Allan y Jhon}

\begin{document}

\maketitle

\section{Codificación}
En presente trabajo si se quiere usar el modo multiCAPTCHA se usa el esquema de Secreto Compartido.
El algoritmo de secreto compartido se realiza con enteros y no con cadenas de caracteres, por lo que es necesario convertir la cadena de caracteres a un entero.
Como se explica a continuacón.\\\\

Se tiene un conjunto de caracteres $AL$ compuesto por $AL=\{A,B,...,Z\}\cup\{a,b,...,z\}\cup\{0,1,...,9\}\cup\{+,/\}$ con una cardinalidad $|AL|=64$.

Para obtener una representación binaria de 64 elementos son necesarios 6 bits por lo que para todos los elementos $\sigma\epsilon AL$ existe una representación binaria. Una vez establecido esto el procedimiento para realizar la conversión es el siguiente:
\begin{enumerate}
 \item Tomamos una cadena de caracteres y la separamos caracter por caracter y los intercambiamos por su correspondiente número entero en $AL$ $\alpha _0||\alpha _1||...||\alpha _m$
 \item Posteriormente cada uno de los enteros lo convertimos en un binario de 6 bits y se concatenan uno detrás del otro $\Psi\longleftarrow bin_6(\alpha _0)||bin_6(\alpha _1)||...||bin_6(\alpha _m$
 \item La cadena binaria $\Psi$ la convertimos a entero $v\longleftarrow toInt(\Psi)$
\end{enumerate}
El entero $v$ que obtenemos es el valor que usaremos en el algoritmo de secreto compartido.
\\
\section{Ejemplo}
Tenemos la cadena $STR=`ABC`$ de la cual cambiaremos cada caracter por su correspondiente valor entero en $AL$ quedando de la siguiente manera $\alpha =\{0,1,2\}$
\\ 
\\
Ahora cada  uno de los elementos de $\alpha$ lo convertiremos a su correspondiente representacion binaria, $bin_6(0)=000000, bin_6(1)=000001, bin_6(2)=000010$ y concatenamos cada una quedando $\Psi = 000000000001000010$
\\
\\
la cadena binaria $\Psi$ se convertirá en un entero $v=toInt(\Psi )$ que da como resultado $v=66$

\section{Decodificación}
Tambien es necesario convertir un entero a una cadena de caracteres y para esto se realiza el proceso inverso:
\begin{enumerate}
 \item El entero $v$ es convertido en un número binario  $z=toBin_6(v)$ 
 \item Separamo $z$ en cadenas de 6 bits y cada una de ellas la interpretamos como un entero $toInt(z_0)||toInt(z_1)||...||toInt(z_w)$
 \item Cada uno de estos valores son convertidos a su correspondiente caracter en $AL$ y concatenados para generar la cadena de caracteres final.
\end{enumerate}

\section{Ejemplo}
El entero $v=66$ se representa como una cadena de 18 bits $z=000000000001000010$, la cual se divide en sub cadenas 6 bits quedando $z_0=000000, z_1=000001, z_2=000010$, para cada uno de estos números binarios se procede a convertirlo en un entero $toInt(z_0)=0, toInt(z_1)=1, toInt(z_2)=2$, por último estos son intercambiados por sus correspondientes caracteres en $AL$ y concatenados resultando en $s=`ABC`$

\end{document}
