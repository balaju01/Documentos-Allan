\documentclass[a4paper,12pt]{article}
\usepackage{graphicx}
\usepackage{cite}
\usepackage{hyperref}

\title{La Guitarra panzona}
\subtitle{Un estudio desde el diseño sonoro}
\author{Allan Ulises Zepeda Ibarra}
\date{\today}

\begin{document}

\maketitle

\begin{abstract}
Aquí va un breve resumen del artículo, explicando su objetivo y enfoque.
\end{abstract}

\section{Introducción y contexto}
Presentación del instrumento analizado. Breve explicación del enfoque del estudio e importancia en su contexto cultural y musical. \cite{Vaca2008-oi}

\section{Agradecimientos y dedicatoria}
Reconocimiento a personas clave en la investigación o en la enseñanza de la construcción del instrumento.

\section{Marco Referencial}
Definición del objeto de estudio y explicación de los criterios de análisis del instrumento.

\section{Marco Histórico}
Orígenes del instrumento, influencias culturales e históricas en su evolución y conexiones con otros instrumentos musicales.

\section{Marco Descriptivo}
Descripción detallada del instrumento:
\begin{itemize}
    \item Materiales de construcción.
    \item Variaciones en diseño y tamaño.
    \item Diferencias entre versiones regionales.
\end{itemize}

\section{Modus Operandi (Proceso de construcción)}
Técnicas tradicionales de fabricación, herramientas y materiales utilizados, innovaciones en la construcción y su impacto en la sonoridad.

\section{Heurística (Experimentación y mejoras en la fabricación)}
Métodos utilizados para optimizar la acústica y adaptaciones modernas a las técnicas tradicionales.

\section{Función Musical}
Rol del instrumento en su repertorio tradicional, diferentes afinaciones y su impacto en la ejecución, papel dentro de los ensambles musicales.

\section{Conclusiones}
Reflexiones sobre la evolución del instrumento, desafíos en la preservación y fabricación e importancia del conocimiento tradicional en la laudería.

\section{Referencias y Bibliografía}
\bibliographystyle{plain}
\bibliography{referencias}  % Crear un archivo referencias.bib con las fuentes necesarias.

\end{document}