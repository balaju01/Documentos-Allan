\documentclass{article}
\usepackage[a4paper,margin=1in]{geometry}
\usepackage{hyperref}
\begin{document}

\title{Currículum Vitae Artístico}
\author{Allan Ulises Zepeda Ibarra}
\date{}
\maketitle

\section*{Datos Personales}
\begin{itemize}
    \item Nombre artístico: Allan Ibarra
    \item Fecha de nacimiento: 31 de julio de 1992
    \item Correo electrónico: \href{mailto:balaju01@gmail.com}{balaju01@gmail.com}
    \item Teléfono: 5534705635
\end{itemize}

\section*{Perfil}
Músico, laudero, productor e investigador especializado en la música tradicional mexicana y latinoamericana. Con una trayectoria destacada en la interpretación, construcción y restauración de instrumentos tradicionales, así como en la investigación y transcripción de repertorios musicales. Ha participado en festivales nacionales e internacionales, colaborando con reconocidos ensambles y artistas. Actualmente cursa la Licenciatura en Composición, Producción y Gestión Cultural en el Colegio Andrew Bell.

\section*{Formación Académica y Artística}
\begin{itemize}
    \item \textbf{Licenciatura en Composición, Producción y Gestión Cultural} (En curso, 2023 - actualidad)\\ Colegio Andrew Bell, Veracruz, México
    \item \textbf{Técnico Instrumentista} (2013)\\ CEDART Luis Spota Saavedra
    \item \textbf{Diplomado en Ensambles Musicales} (2012)\\ INBA Subdirección General de Educación e Investigación Artística
    \item \textbf{Talleres Especializados:}
    \begin{itemize}
        \item Diplomado en Estructuras Musicales para el Jarabe (Música y Baile Tradicional A.C., 2019)
        \item El sistema músico dancístico de la Tierra Caliente (David Durán Naquid, 2020)
        \item Magos de la rima (Ana Zarina Palafox Méndez, 2021)
        \item Marquetería y barniz para instrumentos (Flor Centurión, 2024)
        \item Incrustación en madera y taracea (Flor Centurión, 2024)
    \end{itemize}
\end{itemize}

\section*{Experiencia Artística}
\subsection*{Presentaciones Destacadas (2019-2024)}
\begin{itemize}
    \item Gira musical por Europa promoviendo la música tradicional mexicana con el grupo Sonalli (2019)
    \item Festival de Folklor en Alemania, Francia y Polonia con el grupo Sonalli (2019)
    \item Festival Vida y Son con Trío Cultura Tradicional (Tixtla, Guerrero, 2023)
    \item Verbena Navideña en el Zócalo con ChanequeSon (CDMX, 2023)
    \item XIX Foro de Música Tradicional de la Fonoteca INAH con ChanequeSon (CDMX, 2023)
    \item Festival Son para Milo con ChanequeSon (CDMX, 2023)
    \item Feria del Libro Azcapotzalco con Los Folkloristas (CDMX, 2024)
    \item Feria del Huapango de Amatlán con Trío Cultura Tradicional (Veracruz, 2024)
    \item Festival Cultural Tarandacuao con ChanequeSon (Guanajuato, 2024)
    \item Congreso Internacional de Arpa con Los Navegantes del Golfo (CDMX, 2024)
    \item Encuentro Nacional de Mariachis Tradicionales con ChanequeSon (Guadalajara, 2024)
\end{itemize}

\subsection*{Colaboraciones y Producción Musical}
\begin{itemize}
    \item \textbf{Músico y productor} en grabaciones de ChanequeSon:
    \begin{itemize}
        \item Disco El Son Redoblado (2022)
        \item Cariño sin Condición (2023)
        \item Son La Galina (2023)
        \item El Toro que Hace Llover (2023)
    \end{itemize}
    \item \textbf{Músico en "Veo"} de Karo Nacho (2022)
\end{itemize}

\section*{Construcción y Restauración de Instrumentos}
\begin{itemize}
    \item Construcción de jarana con técnica hermanada (Cedro y Palo de Rosa, 2024)
    \item Restauración de arpa yaqui (Cambio de clavijero en Roble, Cedro y Encino, 2024)
    \item Construcción de Guitarra panzona (2022)
\end{itemize}

\section*{Docencia y Gestión Cultural}
\begin{itemize}
    \item \textbf{Coordinador} del ciclo de conciertos de ChanequeSon (CDMX, 2024)
    \item \textbf{Tallerista}: Ritmos básicos para la tamborita terracalenteña (CDMX, 2024)
    \item \textbf{Profesor de música} en el proyecto comunitario Jóvenes Orquestas (2014 - 2018)
\end{itemize}

\section*{Investigación y Publicaciones}
\begin{itemize}
    \item \textbf{Libro}: Cuadernos de Música de Don Alberto Albarrán Palacios (Coautor y transcriptor, 2024)
    \item \textbf{Artículo}: Una Pequeña Mirada a la Música y Danza Tradicional de San Felipe del Progreso (Revista Jñatrjo 10, 2025)
\end{itemize}

\section*{Becas y Reconocimientos}
\begin{itemize}
    \item \textbf{Beca Encuentro de las Artes Escénicas (2023)}\\ Programa de Estímulos a la Creación Artística, Secretaría de Cultura Federal
\end{itemize}

\section*{Colaboraciones Recientes}
\begin{itemize}
    \item Grupo Sonalli (2024)
    \item Grupo Los Mañaneros (2024)
    \item Los Folkloristas (2024)
\end{itemize}

\section*{Contacto}
Correo electrónico: \href{mailto:balaju01@gmail.com}{balaju01@gmail.com}\\
Teléfono: 5534705635

\end{document}

