\documentclass[a4paper,10pt]{article}


%opening
\title{Edward Fredkin}
\author{Zepeda Ibarra Allan Ulises}

\begin{document}

\maketitle

Nacido el 1 de enero de 1934 es un programador de computadoras, un piloto, asesora a empresas y gobiernos y un f\'{\i}sico. 
Sus intereses principales se refieren a la computadora digital como modelos de los procesos b\'asicos de la f\'{\i}sica. 
Sus contribuciones principales incluyen su trabajo sobre la computaci\'on reversible y aut\'omatas celulares.
Se menciona la importancia de la computaci\'on reversible, la puerta Fredkin representaba un avance esencial. 
Fredkin escribi\'o un ensamblador llamado FRAP (tambi\'en llamado programa de montaje de Fredkin) y el primer sistema operativo. 
 \'El invent\'o y dise\~n\'o el primer sistema de interrupción moderna, llamado "Secuencia Break".
\'El es el inventor de la estructura de datos trie. El ordenador de Billar-Ball para la computaci\'on reversible.
En 1984, fue galardonado con el 'Premio Dickson de la Ciencia'.

\end{document}
