\documentclass[a4paper,12pt]{article}


%opening
\title{Macy Conference}
\author{Arcos Ayala Jonathan}

\begin{document}

\maketitle

\section{¿Que es la Macy Conference?}

Las Macy Conferences eran un conjunto de reuniones de acad\'emicos de diversas disciplinas, 
celebrada en Nueva York bajo la direcci\'on de Frank Fremont-Smith comenzando en 1941 y terminando en 1960. 
El objetivo expl\'{\i}cito de las conferencias fue promover comunicaci\'on significativa en todas las disciplinas cient\'{\i}ficas, 
y restaurar la unidad de la ciencia.\\ Los temas tratados en diferentes series de conferencias incluyen: envejecimiento, adrenal corteza, 
antioxidantes biol\'ogicos, coagulaci\'on de la sangre, la presi\'on arterial, los tejidos conectivos, la infancia y la ni\~nez, 
lesi\'on hep\'atica, interrelaciones metab\'olicas, impulso nervioso, problemas de conciencia, y la funci\'on renal.

\section{Macy Cybernetics Conferences}
Las Cybernetics Conferences se celebraron entre 1946 y 1953, organizado por la Fundaci\'on Josiah Macy, Jr. ,
 motivado por Lawrence K. Frank y Frank Fremont-Smith de la Fundación.
\\
Los Cybernetics Conferences fueron particularmente compleja como resultado de reunir el grupo m\'as diverso de participantes de 
cualquiera de las Macy Conferences, por lo que fueron los m\'as dif\'{i}ciles de organizar y mantener .
\\
El prop\'osito principal de esta serie de conferencias fue establecer las bases de una ciencia general del funcionamiento de la mente 
humana. Generando grandes avances en la teor\'{i}a de sistemas , la cibern\'etica , y lo que m\'as tarde que se conoce como la ciencia cognitiva .
\\
Primera Cybernetic Conference, 21 a 22 marzo 1946\\
Segunda Cybernetic Conference, 17 hasta 18 oct 1946\\
Tercera Cybernetic Conference, marzo 13 hasta 14 1947\\
Cuarta Cybernetic Conference, 23-24 octubre, 1947\\
Quinta Cybernetic Conference, 18 a 19 marzo, 1948\\
Sexta Cybernetic Conference, 24 a 25 marzo 1949\\
Séptima Cybernetic Conference, 23 a 24 marzo 1950\\
Octava Cybernetic Conference, 15 a 16 marzo, 1951\\
Novena Cybernetic Conference, 20 al 21 marzo 1952\\
Décima Cybernetic Conference, 22 a 23 marzo 1953\\
                                                                                                                                                                                                                                                                       
\section{Integrantes de la Macy Conference}
A lo largo de las 10 macy conference's realizadas estos son algunos de los participantes de estas.
\begin{itemize}
  \item Lawrence K. Frank {\itshape Ciencias Sociales}
  \item Warren McCulloch {\itshape Neuro Psycriatia}
  \item Margaret Mead {\itshape Antropologia}
  \item Gregory Bateson {\itshape Antropologia}
  \item Arturo Rosenblueth {\itshape Psicologia}
  \item Julian Bigelow {\itshape Ingenieria Electrica}
  \item Rafael Lorente de N\'o {\itshape Neuro Physiologia}
  \item Walter Pitts {\itshape Matematicas}
  \item Ralp W. Gerard {\itshape Neuro Physiologia}
  \item George Evelyn Hutchinson {\itshape Ecologia}
  \item Heinrich Kl\"uver {\itshape Psicologia}
  \item Paul Lazarsfeld {\itshape Sociologia}
  \item Kurt Lewin {\itshape Psicologia}
  \item John Von Neumann {\itshape Matematicas}
  \item Leonard J. Savage {\itshape Matematicas}
  \item Norbert Wiener {\itshape Matematicas}
  \item	Henry Brosin {\itshape Psiquiatria}
  \item Heinz Von Foerster {\itshape Ingenieria Electrica}
  \item Theodore Schneirla {\itshape Psicologia Comparativa}
  \item Hans Lukas Teuber {\itshape Psicologia}
  \item Harold Abramson {\itshape Medicina} 
  \item Nathan Ackerman {\itshape Psiquiatria} 
  \item Vahe Amassian  {\itshape Neuro psicologia} 
  \item W. Ross Ashby {\itshape Psycriatia} 
  \item Yehoshua Bar-Hillel {\itshape Logica Matematica} 
  \item Morris Bender {\itshape Neuro psicologia} 
  \item Herbert Birch {\itshape Psicologia Animal} 
  \item John Bowman {\itshape Sociologia} 
  \item Frederick Bremer {\itshape Neuro psicologia} 
  \item Yuen Ren Chao {\itshape Ling\"uistica} 
  \item Eilhardt von Domarus {\itshape Neuro psyquiatria} 
  \item Max Delbr\"uck {\itshape Biofisica} 
  \item Jan Droogleever-Fortuyn {\itshape Neuro psicologia} 
  \item Erik Erikson {\itshape Psicoanalisis} 
  \item Leon Festinger {\itshape Psicologia Social}
  \item Frederick Fitch {\itshape Logica} 
  \item Roman Jakobson {\itshape Ling\"uistica} 
  \item Clyde Kluckhohn {\itshape Antropologia} 
  \item Wolfgang K\"ohler {\itshape Psicologia} 
  \item Dorothy Lee {\itshape Antropologia} 
  \item Joseph Licklider {\itshape Psicologia} 
  \item Howard Liddell {\itshape Psicologia} 
  \item Donald Lindsley {\itshape Psicologia} 
  \item William Livingston {\itshape Medicina} 
  \item David Lloyd {\itshape Neuro Sociologia} 
  \item John Lotz {\itshape Ling\"istica} 
  \item Duncan Luce {\itshape Psicologia} 
  \item Donald MacKay {\itshape Fisica} 
  \item Turner McLardy {\itshape Neuro psyquiatria} 
  \item Frederick Mettler {\itshape Anatomia}
  \item Marcel Monnier {\itshape Medicina} 
  \item Charles Morris {\itshape Linguistica} 
  
\end{itemize}

\end{document}
