\documentclass[a4paper,10pt]{article}


%opening
\title{Complejidad y el problema de la criminalidad}

\author{Allan Ulises Zepeda Ibarra}

\begin{document}

\maketitle
El Mtro. Jorge Jim\'enez Zamudio hablo acerca del problema social de la criminalidad y que siempre ha sido una curiosidad querer predecir cu\'ando puede 
pasar un crimen y de esta forma prevenir que esto pase o en su defecto atrapar al responsable infraganti para llevarlo ante la ley. 
El Mtro. Propone a trav\'es del an\'alisis de algunas variables como son el infractor, la víctima, el momento en que sucede el delito, 
y a trav\'es de estas variables propone un sistema de tres ecuaciones diferenciales no lineales en las que se describe el comportamiento 
del crimen, posteriormente grafica los valores arrojados por este sistema de ecuaciones, al momento de graficar arroja una gr\'afica de 
c\'{\i}rculos conc\'entricos.La Mtra. Jeanett L\'opez Garc\'{\i}a habla acerca los fractales, los atractores y la teor\'{\i}a del caos. Primeramente explico un poco acerca de lo que son los atractores
 que son puntos a los que se va acercando la informaci\'on y forma patrones, el trabajo de la Dra. se enfoca en funciones iterativas que 
se bifurcan obteniendo dos caminos posibles por medio de probabilidades y usando ecuaciones que se basan en la teor\'{\i}a del caos ya que al 
modificar levemente una variable al inicio de la funci\'on podemos obtener un resultado distinto, al graficar despu\'es de vario siclos 
obtenemos un fractal. La conclucion es que si pudi\'eramos predecir todo el crimen y pudi\'eramos 
atrapar a todos los maleantes enseguida al otro d\'{\i}a abr\'{\i}a m\'as maleantes.



\end{document}
