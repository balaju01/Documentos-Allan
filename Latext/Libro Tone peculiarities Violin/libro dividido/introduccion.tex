\chapter*{Introducción}
\addcontentsline{toc}{chapter}{INTRODUCCIÓN}
Estas conferencias, dirigidas a una audiencia imaginaria de estudiantes de violín, fueron originalmente escritas y parcialmente publicadas en el Western Musician, Dixon, Illinois, para el entretenimiento de los numerosos lectores de esta revista musical. Dos de las conferencias ahora aparecen impresas por primera vez. Como se empleó un estilo familiar, se evitaron términos técnicos abstractos en la medida de lo posible sin interferir con la claridad y precisión.

Los experimentos, resultados y conclusiones, tal como se registran aquí, no son fantasías de la imaginación, como podría inferirse al principio, sino que son conclusiones obtenidas a través de experimentos prácticos, y también por accidentes ocurridos en mi experiencia.

Así, cuando pacientes violines llegaron a mi hospital, me sentí feliz, y debido a mi entusiasta devoción a los problemas de diagnóstico tonal, trabajé sobre ellos, y sobre ellos, hasta declararlos curados o incurables. Algunos de esos pacientes violines eran, como algunos pacientes humanos, bendecidos con buenas constituciones inherentes desde el principio, y eran capaces de recibir valores tonales mejorados a partir del ajuste cuidadoso de los factores modificadores del tono, mientras que otros eran tan inherentemente malos desde el día en que fueron llamados "violín" (mal llamados), que solo heredaron un tono ruidoso; sin embargo, el tono ruidoso hizo "casos interesantes" de esta última clase debido a que ofrecían razones incontrovertibles para el tono inferior, razones que demostraban concluyentemente la verdad en la afirmación "sin material superior, sin violín superior."

Durante mi periodo de trabajo activo, siempre tuve en mente las siguientes preguntas:
\begin{itemize}
    \item "¿Cómo opera el violín para producir sonido musical?"
    \item "¿Qué agentes, conectados con el violín, operan para modificar el tono?"
    \item "¿Cuáles son las causas del tono inferior en un violín?"
    \item "¿Cuáles son las causas del tono superior en un violín?"
\end{itemize}

Algunas de estas preguntas las he resuelto a mi satisfacción, pero no pretendo que tales soluciones sean aceptables para otros estudiantes de fenómenos tonales del violín; ni pretendo que todos esos problemas de tono hayan recibido solución. Algunas de mis conclusiones están en desacuerdo con las conclusiones de investigadores científicos notables, pero no reclamo infalibilidad para mis propias conclusiones. Errar es humano. Seguir el error también es humano. Así, seguí una conclusión científica sobre la producción y modificación del tono del violín que requirió experiencias de veinticinco años para disipar la ilusión. Sobre esta base, se advierte al estudiante de violín sobre el peligro de seguir teorías abstractas bajo el disfraz de la ciencia.

Creo que las teorías, incluso cuando se basan en demostraciones prácticas repetidas en varios violines, deben presentarse solo como conclusiones de un individuo que intenta resolver un problema en el que la acción caprichosa de la madera ha sido, es y siempre puede seguir siendo una cantidad desconocida; y presento la idea de que tal cantidad desconocida es la razón por la cual la ciencia fracasa al intentar construir un violín por encargo.

Los siguientes problemas permanecen sin elucidación:
\begin{itemize}
    \item "Acción caprichosa inherente de la madera."
    \item "Diferentes grados de concentración de ondas sonoras en las salidas según diferentes grados de arqueado de las placas."
    \item "El fenómeno de elevar la altura tonal al agrandar el área de las salidas."
\end{itemize}

Se presenta la opinión de que las soluciones para los dos primeros problemas pondrán la calidad tonal del violín bajo el control de la voluntad. No obstante, a pesar de las dudas de resolver los problemas involucrados en la acción caprichosa de la madera, el valor de tal solución sigue siendo un incentivo poderoso para continuar el esfuerzo. El deseo de violines que posean un tono "rico" combinado con una marcada intensidad de tono es un estímulo que supera el estímulo del oro fino; y quien descubra un método para producir tales violines a voluntad se convertirá en un rey en su propio derecho.

Mi método para llegar a conclusiones sobre la potencia de cada modificador del tono del violín es investigar las causas del tono ruidoso, tono dulce, tono poderoso, tono hueco, tono fino, tono "todo por dentro", tono "todo por fuera", volumen de tono, intensidad del tono, altura tonal, tonos dobles no musicales, tonos abiertos poderosos con tonos altísimos débiles, tonos resultantes o armónicos a bassa, sobretonos consonantes, sobretonos disonantes, el "tono rico", el "tono frío", tono simpático, uniformidad del poder tonal, y carácter tonal basado en el carácter tonal de la voz humana.

En este trabajo, las conclusiones aquí presentadas siguen experimentos realizados tanto en violines antiguos como nuevos, y el número de tales violines asciende a cientos. A partir de las deducciones así obtenidas, mi deseo es dar prominencia a las siguientes proposiciones:
\begin{itemize}
 \item Las peculiaridades tonales que existen en un violín dado pueden no existir en ningún otro violín.
 \item Escribir sobre peculiaridades tonales que existen en un violín dado como necesidades infalibles para todos los violines es engañoso.
 \item Encontrar dos violines que posean valores tonales precisamente similares es igualmente difícil que encontrar dos voces que posean valores tonales precisamente similares.
 \item Ningún fabricante de violines, sea quien sea, ha sido capaz de otorgar un valor tonal destacado a cada violín.
 \item Que el pastor de ovejas de la montaña puede producir un violín con valores tonales iguales a los mejores.
 \item Que el mecánico hábil, guiado por un instinto musical infalible, produce un número vastamente mayor de violines superiores que el mecánico sin tal instinto.
 \item Que todos los fabricantes de violines pueden experimentar derrotas ocasionales.
 \item Que, salvo accidente, el violín superior es producto de una habilidad mecánica superior combinada con un sentido musical superior, todo dirigido sobre material superior.
\end{itemize}
No parece haber otro método que ofrezca un valor igual a las conclusiones que el método aquí presentado para determinar la potencia y operación de cada factor que interviene en la producción y modificación del tono del violín.

A la evaluación de tales factores he dedicado una vida; no en teorías abstractas, sino sentado en el banco mientras repetía demostración tras demostración, año tras año, década tras década, desde la juventud hasta la vejez, decidido a aislar, evaluar y conocer la operación de todos y cada uno de los factores subyacentes en los fenómenos tonales del violín, o morir en el intento. A los sesenta y tres años, la muerte estuvo cerca, y tres años después siguió cerca, dejando solo mi brazo derecho suficientemente útil para guiar la pluma. Ahora es seguro que no alcanzaré la meta de mi ambición.

Bajo tales dificultades, escribir es laborioso; además, el material aquí presentado se compone completamente de memoria, no se tomaron notas con vistas a la publicación. En el momento presente, la conservación necesaria de fuerzas me limita a un período diario limitado de trabajo; por lo tanto, abandono la reescritura planeada de la publicación preliminar, de la cual se hicieron las correcciones necesarias, y de la cual se omiten algunos párrafos, y a la cual se agregan las conferencias xvi y xvii. Esta publicación se presenta como mi legado tanto para el estudiante de violín como para el fabricante de violines estadounidense. Que el siguiente registro se reduzca a la escritura y se publique es algo debido enteramente al estímulo ofrecido por un fabricante de violines moderno; por lo tanto, cualquier entretenimiento o cualquier otro valor que se pueda encontrar en estas páginas es algo no atribuible únicamente al coraje de, FREDERICK CASTLE. Lowell, Indiana, 20 de marzo de 1906.