\chapter*{Introducción}
\addcontentsline{toc}{chapter}{INTRODUCCIÓN}
Estas conferencias, dirigidas a un público imaginario de estudiantes de violín, fueron escritas originalmente para el entretenimiento de los lectores de la revista \emph{Western Musician}. Se publican ahora dos de las conferencias por primera vez. Se empleó un estilo familiar, evitando en la medida de lo posible los términos técnicos abstrusos sin interferir con la claridad y la precisión.

Los experimentos, resultados y conclusiones aquí registrados no son romances de la imaginación, sino conclusiones obtenidas mediante experimentos prácticos y accidentes ocurridos en mi experiencia. 

Así, cuando los pacientes de violín llegaban a mi hospital, yo era feliz, y debido a mi devoción entusiasta a los problemas de diagnóstico de tono, trabajaba en ellos y sobre ellos hasta declararlos curados o incurables. Algunos de esos pacientes de violín eran como algunos pacientes humanos, bendecidos con constituciones inherentemente buenas para empezar, y eran capaces de recibir valores de tono mejorados mediante un ajuste cuidadoso de los factores modificadores de tono, mientras que algunos de ellos eran tan inherentemente malos desde el día en que fueron nombrados ``violín`` (mal nombrados), que solo el tono ruidoso era su herencia; sin embargo, el tono ruidoso creo ``casos interesantes`` de esta última clase debido a que ofrecía razones incontestables para el tono inferior, razones que demostraban de manera concluyente la verdad en la afirmación, ``sin material superior, sin violín superior``.

Durante mi periodo de trabajo activo, siempre tuve en mente las siguientes preguntas:
\begin{itemize}
    \item ¿Cómo opera el violín para producir sonido musical?
    \item ¿Qué agentes, conectados con el violín, operan para modificar el tono?
    \item ¿Cuáles son las causas del tono inferior en un violín?
    \item ¿Cuáles son las causas del tono superior en un violín?
\end{itemize}

Algunas de estas preguntas las he resuelto a mi propia satisfacción, pero no se afirma que tales soluciones serán aceptables para otros estudiantes de fenómenos de tono del violín; ni se afirma que todos esos problemas de tono han recibido solución. Algunas de mis conclusiones están en desacuerdo con las conclusiones de destacados investigadores científicos, pero para mis propias conclusiones, no se reclama infalibilidad. Errar es humano. Seguir el error también es humano. Así, seguí una conclusión científica sobre la producción y modificación del tono del violín que requirió experiencias de veinticinco años para disipar el engaño. Sobre esta base, se advierte al estudiante de violín del peligro de seguir la teoría abstracta bajo la apariencia de ciencia. Creo que las teorías, incluso cuando se basan en demostraciones prácticas repetidas a menudo en varios violines, deben presentarse solo como conclusiones de un individuo que intenta resolver un problema en el que la acción caprichosa de la madera siempre ha sido, es ahora y siempre podrá seguir siendo una cantidad desconocida; y presento la idea de que tal cantidad desconocida es la razón por la cual la ciencia se encuentra con la derrota al intentar construir un violín a pedido.




Los siguientes problemas permanecen sin elucidación:
\begin{itemize}
    \item ``Acción caprichosa inherente de la madera.``
    \item ``Diferentes grados de concentración de ondas sonoras en las salidas según diferentes grados de arqueado de las placas.``
    \item ``El fenómeno de elevar la altura tonal al agrandar el área de las salidas.``
\end{itemize}

Se presenta la opinión de que las soluciones para los dos primeros de estos problemas pondrán la calidad del tono del violín al mando de la voluntad. A pesar de las dudas sobre la resolución de los problemas involucrados en la acción caprichosa de la madera, el valor de tal solución sigue siendo un poderoso incentivo para el esfuerzo continuo. El deseo de violines que posean un tono ``rico`` combinado con una marcada intensidad de tono es un estímulo que supera el estímulo del oro fino; y quien descubra un método para producir tales violines a voluntad se convertirá en un rey por derecho propio.

Mi método para llegar a conclusiones sobre la potencia de cada modificador del tono del violín es investigar las causas del tono ruidoso, el tono dulce, el tono potente, el tono hueco, el tono delgado, el tono ``todo adentro``, el tono ``todo afuera``, el volumen del tono, la intensidad del tono, el tono, los tonos de doble parada no musicales, los potentes tonos abiertos con tonos altísimos débiles, los tonos resultantes o armónicos a bassa, los sobretonos consonantes, los sobretonos disonantes, el tono ``rico``, el tono ``frío``, el tono simpático, la uniformidad de la potencia del tono y el carácter del tono basado en el carácter del tono de la voz humana.

En este trabajo, las conclusiones aquí presentadas siguen a experimentos dirigidos tanto en violines antiguos como nuevos, y el número de tales violines asciende a cientos. De las deducciones así obtenidas, es mi deseo dar prominencia a las siguientes proposiciones:

\begin{enumerate}
 \item Las peculiaridades tonales que existen en un violín dado pueden no existir en ningún otro violín.
 \item Escribir sobre peculiaridades tonales que existen en un violín dado como necesidades infalibles para todos los violines es engañoso.
 \item Encontrar dos violines que posean valores tonales precisamente similares es igualmente difícil que encontrar dos voces que posean valores tonales precisamente similares.
 \item Ningún fabricante de violines, sea quien sea, ha sido capaz de otorgar un valor tonal destacado a cada violín.
 \item Que el pastor de ovejas de la montaña puede producir un violín con valores tonales iguales a los mejores.
 \item Que el mecánico hábil, guiado por un instinto musical infalible, produce un número vastamente mayor de violines superiores que el mecánico sin tal instinto.
 \item Que todos los fabricantes de violines pueden experimentar derrotas ocasionales.
 \item Que, salvo accidente, el violín superior es producto de una habilidad mecánica superior combinada con un sentido musical superior, todo dirigido sobre material superior.
\end{enumerate}
Que el método aquí presentado para determinar la potencia y el funcionamiento de cada factor que interviene en la producción y modificación del tono del violín, no parece haber otro método que ofrezca igual valor a las conclusiones.

A la ponderación de tales factores, he dedicado toda una vida; no en teorizar abstracto, sino en sentarme en el banco mientras repito demostración tras demostración, año tras año, década tras década, desde la juventud hasta la vejez, decidido a aislar, ponderar y conocer el funcionamiento de todos y cada uno de los factores subyacentes a los fenómenos del tono del violín o morir en el intento. A los sesenta y tres años, la muerte se acercó, y tres años después permaneció cerca, dejando solo mi brazo derecho lo suficientemente útil para guiar la pluma. Ahora es seguro que no alcanzaré la meta de mi ambición.

Bajo tales dificultades, la escritura es laboriosa, además, el asunto aquí presente se compone enteramente de memoria, sin que se hayan tomado notas con miras a la publicación. En el momento actual, la conservación necesaria de la fuerza me confina a un período diario limitado para el trabajo; de ahí el abandono de la reescritura prevista de la publicación preliminar, de la cual se realizan las correcciones necesarias, y de la cual se omiten algunos párrafos, y a la cual se añaden las conferencias xvi y xvii. Esta publicación se presenta como mi legado tanto al estudiante de violín como al fabricante de violines estadounidense. Que el siguiente registro se reduzca a la escritura y se publique es un asunto totalmente debido al estímulo ofrecido por un fabricante de violines moderno; por lo tanto, cualquier entretenimiento, o cualquier otro valor que se pueda encontrar en estas páginas es algo no atribuible solo al coraje de FREDERICK CASTLE.

Lowell, Indiana, 20 de marzo de 1906.
