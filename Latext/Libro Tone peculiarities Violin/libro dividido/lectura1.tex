\chapter*{Conferencia I}
\addcontentsline{toc}{chapter}{LECTURA I}
SEÑORES ESTUDIANTES DEL VIOLÍN, en esta, nuestra primera sesión, aprovecho la oportunidad para ofrecerles mis felicitaciones por los siguientes hechos interesantes. Primero: se han descubierto las causas del tono ruidoso en los violines. Segundo: se ha perfeccionado una forma exitosa de preservar las superficies interiores del violín de la desintegración por el calor y la humedad. Tercero: las áreas de la tapa armonica del violín, responsables de la producción y aumento del tono, han sido localizadas y definidas. Cuarto: se ha demostrado un método de graduación de la tapa armonica que asegura la máxima uniformidad en el poder tonal. Quinto: los principios que rigen la intensidad del tono del violín han salido a la luz. Sexto: los principios que gobiernan el poder del tono del violín se han expresado en palabras. Séptimo: se ha descrito la calidad de la madera de la tapa armonica que permite obtener un ``tono rico`` en el violín. Octavo: se ha registrado el poder del accidente para disipar la oscuridad y la ilusión. Noveno: algunas conclusiones científicas sobre ``cómo opera el violín para producir sonido musical`` han sido sacudidas. Décimo: se ha demostrado la falacia en la afirmación de que ``los mejores Cremonas son vehículos necesarios para la interpretación de las partituras de Haydn, Mozart y Beethoven``. Undécimo: se ha intentado corregir los agravios impuestos al fabricante de violines moderno por el ``comerciantes de violines antiguos``.

Durante nuestro curso de estudio, se les presentarán algunas ideas sobre el tono del violín que hasta ahora no se han expresado. De hecho, les prometo que habrá muy poco contenido repetido en nuestro programa. Como no es mi intención privarlos del placer que ofrece la anticipación, solo se les darán pequeñas dosis a la vez. Este plan se adopta para evitar dañar su capacidad de digestión y para asegurar su asistencia regular.

El gusto por el tono varía, varía a través de cada grado de cultura musical. Un tono que agrada a una persona puede no agradar a otra. Ningún intérprete puede tocar al máximo en un instrumento cuyo tono le resulte desagradable. Afortunadamente, el violín ofrece una variedad de calidad tonal tan infinitamente grande que cada violinista en la Tierra puede poseer uno con una calidad tonal que se ajuste a su gusto.

Uno podría pensar que es posible fabricar violines con un estándar único de calidad tonal, pero el hecho es que la peculiaridad inherente de la acción de la madera lo impide.

Existe algo que se aproxima a un estándar tonal invariable para toda la gama de instrumentos de viento e instrumentos de percusión.

Pero la calidad tonal invariable se detiene abruptamente ante la presencia de la familia del violín. Podemos imaginar la inmensa sorpresa de alguien que nunca ha escuchado otros instrumentos que no sean de viento al ser introducido a esta familia de violines. A medida que toma violín tras violín, viola tras viola, violonchelo tras violonchelo, no encuentra dos que posean un carácter tonal idéntico. Cada violín, cada viola, cada violonchelo tiene una calidad tonal peculiar a sí mismo. Estas peculiaridades son tan marcadas que pronto se vuelve capaz de nombrar cada violín con cuyos tonos está familiarizado, aunque esté con los ojos vendados o en una habitación distante, nombrándolos con la misma certeza con la que puede identificar diferentes cantantes con cuyos tonos está familiarizado.

Se vuelve curioso por conocer la razón o las razones de la infinita peculiaridad tonal de ese maravilloso instrumento musical llamado ``violín``.

Por observación, descubre que las cuerdas G y D a veces poseen un carácter tonal grave; en otros momentos, estas cuerdas poseen un carácter barítono-tenor; en algunos casos, las cuerdas A y E poseen un carácter mezzo-soprano; en otros casos, las cuerdas A y E poseen únicamente un carácter soprano.

Debido a estas peculiaridades tonales, clasifica los violines en cuatro categorías, de la siguiente manera:
\begin{itemize}
 \item Bajo-mezzo-soprano.
 \item Bajo-soprano.
 \item Barítono-mezzo-soprano.
 \item Barítono-soprano.
\end{itemize}
Por experimentación, descubre que estas cuatro clases de carácter tonal pueden ser dadas a los violines a voluntad, y que dependen de diversos grados de espesor de la tapa armonica, junto con modificadores del tono como el tamaño y la posición de las salidas, la capacidad de aire del violín, etc.

Por observación, encuentra un campo de utilidad peculiar en dos de estas clases. Así: El violín con carácter tonal bajo-mezzo-soprano es el instrumento solista más agradable, mientras que el violín con carácter barítono-soprano es decididamente más efectivo para su uso en la orquesta; este último hecho se debe a la alta altura tonal, lo que permite que sus ondas tonales se sitúen sobre las ondas de las demás partes armónicas.

Sorprendentes como son estas peculiaridades, aún encuentra otro hecho en el tono del violín aún más sorprendente; es decir, algunos violines poseen una calidad tonal humana en un grado muy superior a todos los demás dispositivos musicales. Así, se le abre un nuevo mundo de expresión.

Aquí tenemos un dispositivo musical capaz de ``hablar``; capaz de participar en un diálogo al estilo del ``Arkansas Traveler``, un instrumento capaz de detener el canto de las aves silvestres, haciendo que, con el cuello extendido y los ojos iluminados por la maravilla, busquen a ese otro extraño ``cantante`` que emite esos trinos encantadores; un instrumento capaz de estallar en risas alegres al estilo de la partitura de risas en el ``Carnaval`` de Paganini; un instrumento capaz de pronunciar oraciones devotas en la ``Canción sin Palabras`` de Mozart; un instrumento que llena el aire con esos tonos que hacen olvidar los problemas en el ``Sueño`` de Schumann; que despierta la ternura humana con los tonos simpáticos de ``Sweet Home``; que hace llorar a los ojos humanos con esos incomparables, conmovedores y desesperados tonos de despedida, como la valiente, amorosa e inquebrantable Norma, condenada a muerte en la hoguera por su severo padre druida, cantando en el ``Duetto e Scena Ultima`` mientras asciende a esa pira funeraria ardiente—¡Dios mío! ¡Qué lágrimas cegadoras! ¡Qué agonía!

En todo el amplio mundo, no hay ningún instrumento musical que se acerque al violín. Nuestro investigador del violín se ha convertido ahora en un devoto del violín. Los tonos encantadores de este prodigio afinado a la humanidad lo obligan a inclinarse y adorarlo como ``El Rey``.
\begin{itemize}
 \item ¡Tú, oh violín!
 \item ¡Tú, que sonríes tanto al mendigo como al rey!
 \item Tú, cosa que ríes, lloras, rezas, cantas!
 \item ¡Tú, cosa de belleza!
 \item ¡Tú, alegría eterna!
 \item Ni reyes, ni reinas, ni potentados
 \item Reinan con tu absolutismo.
 \item ¡Tú, oh violín!
\end{itemize}
¿Condenas a este hombre por tal adoración idólatra?

He dedicado más de cincuenta años a la búsqueda de las causas de las peculiaridades tonales del violín; encontrando algunas de ellas, o eso creo. No reclamo un conocimiento superior de la física, ni una penetración superior. Solo reclamo mérito por mi tenacidad.

El fisionomista podría decir de mí: ``tienes una mandíbula cuadrada``. El frenólogo podría decir: ``tienes un desarrollo notable en la región de no-dejar-ir``. Ambos podrían concluir diciendo: ``no tienes nada más digno de mención``. Solo la tenacidad puede mantener a un hombre trabajando durante cincuenta años en un solo problema.

Trabajé cuarenta años intentando que todos los violines tuvieran un tono dulce; en otras palabras, intentando encontrar la causa del ``ruido`` en el tono del violín. Había llegado a la conclusión de que el tono dulce en el violín es un accidente, cuando ocurrió un verdadero accidente que reveló la causa del ruido en menos de diez minutos.

¿Ironía?
Mucha.
Confieso que una solución por accidente es mejor que ninguna solución. Cuando un hombre, incluso con la ayuda de un accidente, vive para demostrar un principio beneficioso para la humanidad, puede partir sabiendo que el mundo es mejor por su existencia.

¿No es un hecho que cuando un hombre puede elevar a la humanidad por encima del ruido desgarrador, desolador y suicida del violín, tiene suficiente base para cualquier reclamo razonable en la Tierra o en el cielo?

Que lo atestigüen millones de personas con ``nervios``.
¡Basta!

Algunos violines tienen un tono dulce; otros no.
La verdad es que pocos violines son realmente dulces; muchos no lo son.
Para el oído atento, la dulzura es el principal elemento de valor en el tono del violín. Curiosamente, hay algunos violinistas que no le dan valor a la dulzura del tono, diciendo: ``me encargaré de la dulzura si logro obtener poder tonal``.

Nunca ha habido un error mayor.
Los mejores violinistas, desde Ole Bull hasta el Sr. ``Sierra-tu-cabeza``, no podían, ni podrán jamás, ocultar el tono ``ruidoso`` de un violín. Admitiendo una diferencia a favor de la técnica hábil con el arco, aun así, por muy hábil que uno sea, el noventa por ciento del público dirá: ``Ese sujeto no sabe tocar el violín``.

Por tono dulce me refiero a un tono no acompañado de ondas sonoras afinadas en claves inarmónicas.

Además, es un error suponer que el ``tono ruidoso`` viaja la misma distancia que el tono dulce. En este punto, la siguiente prueba de alcance ofrece al devoto del tono fuerte una buena oportunidad de desilusión, y también de desprenderse de su riqueza. Es una prueba que he realizado repetidamente, con resultados invariantes.

Como bien saben, un violín de tono fuerte y ruidoso, tocado en una habitación pequeña y desnuda, hace que uno anhele la tranquilidad de un bosque sombreado. De una serie de violines probados en una habitación pequeña, seleccionen el más ruidoso y el más dulce, y llévenlos a un campo abierto, nivelado, que ofrezca al menos 1400 pies lineales de distancia sin obstrucciones. Seleccionen un día sin viento; un día despejado es el mejor, porque cualquier cosa que se acerque a una nube nimbos ayuda mucho en la propagación del sonido. Elijan una hora entre las 10 a. m. y las 4 p. m., ya que en esas horas del día el sonido se propaga con mayor dificultad. Para que el registro tenga valor, lleven un termómetro, un barómetro y un higrómetro, y registren las lecturas de estos instrumentos en el momento de la prueba. Así, el poder de alcance de un violín probado de esta manera puede garantizarse para repetir su rendimiento en cualquier momento bajo condiciones meteorológicas similares. Como saben, las condiciones meteorológicas modifican en gran medida las distancias a las que viaja el sonido. En nuestros meses de verano, estas condiciones a menudo hacen que el tono del violín sea decepcionante. Por lo tanto, cualquier violín puede ser objeto de una crítica tonal injusta.

Al realizar esta prueba de distancia tonal, al menos dos personas deben asistir.
Una tocará una melodía en la cuerda G; la otra se retirará a través del campo hasta una distancia en la que la melodía se distinga débilmente. Esta distancia, medida, se acreditará a esa cuerda. Así se registrará cada cuerda de cada violín.

Esta prueba establece:
\begin{itemize}
 \item Uniformidad del poder tonal.
 \item Intensidad del tono. (Poder de alcance).
 \item Pureza del tono. (Dulzura del tono).
\end{itemize}

En mi experiencia, el tono más dulce viaja invariablemente una mayor distancia. He conocido a violines de tono más dulce que han ganado por 250 pies. En uniformidad del tono, probé un violín antiguo reputado cuyo registro de distancia varía de 1000 pies para la cuerda G a 1480 pies para la cuerda E. Solo he probado dos violines de esta manera, obteniendo un registro de distancia igual para cada cuerda.

Les aseguro que esta prueba de distancia del tono puede causar una profunda sorpresa a los participantes. Yo mismo, después de una larga experiencia, no me atrevo a arriesgarme con el resultado. Esta prueba proporciona una prueba amplia de que el oído atento está en la mejor posición para juzgar el tono.

En este punto presento el ``ruido``. 
Es algo familiar, verdaderamente. 
Es algo que no debería encontrarse en el violín. 
No siempre en los libros de texto encontramos una definición de ``ruido``. 
El ruido parece ser un tema doloroso. La proximidad acentúa su dolor.

La existencia del ``ruido`` es como la densidad de población. Al ruido se le puede achacar la existencia de ``nervios``. Este hecho puede ser probado al retroceder unos siglos, cuando había menos personas en la Tierra, menos cosas en movimiento y muchos menos violinistas, y allí no encontramos registro de ``nervios``. La ciencia nos haría creer que donde no hay oídos, no hay sonido, no hay ``ruido``.

¿Creen en esta historia? Si no desean expresar su incredulidad, al menos pueden llamarla una paradoja. ¡Pero pensar en un lugar donde no haya ruido! ¡Lugar bendito! ¡Debe ser un lugar donde los ángeles no temen pisar! Que no sea invadido por el violín ``ruidoso``, ya que el ruido más desconcertante, el ruido ``le plus terrible``, puede provenir del violín.

¿Qué es el ruido? Si no encuentran una definición que les convenga, lean lo siguiente: El ruido es una agregación de ondas sonoras afinadas en teclas inarmónicas.

La definición misma provoca escalofríos. Estoy orgulloso de ello; del escalofrío, me refiero. El escalofrío es prueba de que mi definición es correcta.

¿Cuál es la causa del ``ruido`` en el tono del violín? Esta pregunta está llena de un interés absorbente para todo el mundo del violín. El accidente que proporciona una solución a esta importante pregunta es tan provocadoramente accidental que me roba todo honor por la solución.

He trabajado toda una vida en esta solución; he leído libros, y libros; he comprado algunos libros por mí mismo; he tomado prestados más. He aprendido allí algunos hechos que requirieron veinte y cinco años para desaprender; había renunciado a la posibilidad de una solución, creyendo que la dulzura del tono del violín era un accidente, cuando ocurrió el siguiente accidente, así: