% Índice personalizado
\chapter*{ÍNDICE}
\addcontentsline{toc}{chapter}{ÍNDICE}
\begin{multicols}{2}
\textbf{ACCIDENTE} No. I, \pageref{accidente}.
\textbf{SONIDO AUDIBLE}, limitado a doce pulgadas de la tapa armónica del violín, \pageref{sonido_audible}.
\textbf{ÁNGULOS}, incidencia y reflexión, \pageref{angulos}.
\textbf{LIBROS}, evidencia poco fiable en, \pageref{libros}.
\textbf{EL MEJOR VIOLÍN SOLISTA}, no el mejor violín de orquesta, \pageref{mejor_violin}.
\textbf{PLANCHA DEL FONDO}, funciones de, \pageref{plancha_fondo}; como agente productor de tono, \pageref{plancha_tono}.
\textbf{CUIDADO DEL BARNIZ}, ejemplo, \pageref{cuidado_barniz}.
\textbf{BARNIZ DE CREMONA}, \pageref{barniz_cremona}.
\textbf{EFECTO TONAL COMBINADO}, media docena de violines de tono uniforme, \pageref{efecto_tonal}.
\textbf{TONO FRÍO}, \pageref{tono_frio}.
\textbf{SOBRETONOS DISONANTES}, causa de, \pageref{sobretonos_dis}.
\textbf{FRAUDE}, \pageref{fraude}.
\textbf{VIOLÍN DE PROPÓSITO GENERAL}, \pageref{violin_general}.
\textbf{GOMAS}, duras, inelásticas, efecto de, \pageref{gomas}.
\textbf{EL CHORRO}, \pageref{chorro}.
\textbf{ARMÓNICOS}, valor de, \pageref{armonicos}.
\textbf{SOBRETONOS ARMÓNICOS}, \pageref{sobretonos_arm}.
\textbf{INTRODUCCIÓN} XII, método para llegar a conclusiones, xiv, ocho proposiciones, xv.
\textbf{INANIDAD}, continuar exigiendo que los solistas solo aparezcan con un Stradivarius o Guarnerius, \pageref{inanidad}.
\textbf{INVITACIÓN}, a mi violín de pastor de ovejas, \pageref{invitacion}.
\textbf{LONGEVIDAD} del violín, 98; ejemplo, 98.
\textbf{PÉRDIDA DE POTENCIA DE TONO}, causas de, 299.
\textbf{CONDICIONES METEÓRICAS}, 23. Sonido musical, ley explícita, 30; que afecta la distancia recorrida por el tono del violín, 280.
\textbf{MÁXIMA UNIFORMIDAD} de la potencia del tono del violín, 205; tonos fundamentales, 208; razones que llevan a un nuevo método para la graduación de la tapa armónica, 214; demostración de las áreas de la tapa armónica que aumentan el tono de cada cuerda, 216; relación para las longitudes de la actividad de la fibra debajo de cada cuerda, 218; tono de concierto hace 200 años, 223; demostración de que los errores en la graduación de la tapa armónica causan una potencia de tono desigual, 225.
\textbf{MÁXIMA POTENCIA DE TONO}, 235; tono "grande", 236; tono del violín separado por dos factores irreconciliables, 237; esteticismo del violín enloquecido, 238; lista de factores que producen la máxima potencia de tono del violín; apariencias físicas de la madera de la tapa armónica que produce la máxima potencia de tono combinada con tono "rico", 246; vibración normal y transversal en la tapa armónica, 249; tono leñoso, 255; dispersión de fuerza, la bola que rebota, 271.
\textbf{RUIDO}, definición de, 25.
\textbf{NODOS}, 294.
\textbf{OH, POSTE}, 276.
\textbf{TONO}, sonido musical, 30.
\textbf{INTÉRPRETES}, opiniones variables de, 36.
\textbf{FILOSOFÍA} involucrada en las condiciones de la superficie interior del violín, 256; lista de principios que modifican el tono del violín, 259; número de cualidades de tono absolutamente al mando del constructor de violines, 261; intensidad del tono del violín producto de cuatro factores, 262; propiedades del aire que afectan la intensidad del tono del violín, 264.
\textbf{PENETRACIÓN} de aceite, 274.
\textbf{PASCUA}, 276.
\textbf{POSTE}, problema de, 301.
\textbf{AJUSTE DEL POSTE}, 303.
\textbf{TONO RICO}, causas de, 96; descripción de la madera que produce tono rico, 97; ilustración, 296.
\textbf{CIENCIA}, nunca hizo un violín, 32.
\textbf{DECLARACIONES CIENTÍFICAS}, recibidas con precaución, 32.
\textbf{MADERA DE LA TAPA ARMÓNICA}, 57; contribución mínima de la ciencia, 58; valor en el escrutinio de la madera de violines usados, 61; contracción desigual, ejemplo, 62; ejemplo, 63; defectos del pino, 66; pino de Michigan, 67; cedro blanco, 69; cambios de color en el pino, 70, acción independiente de fibras contiguas, 71; patético, 78; preservación de superficies interiores, 79; ejemplo de desintegración de la superficie interior, 83.
\textbf{ACCIÓN SIMPÁTICA}, 108; ley de, 109.
\textbf{DULCE VIOLÍN ANTIGUO}, 91.
\textbf{EL REY}, 20.
\textbf{DOS ERRORES}, 22.
\textbf{PRUEBA DE DISTANCIA DE TONO} al aire libre, 23; el oído que escucha en la mejor posición, 24.
\textbf{PROMOTORES COMERCIALES} industria de, 39.
\textbf{HOMBRE DE DOS DÓLARES}, 41.
\textbf{TÚ, VIOLÍN}, 45.
\textbf{TONO}, 45; ocho principios que rigen, 46; regla de aplicación 1, 48; regla 2, 48; regla 3, 49; regla 4, 49; regla 5, 50; regla 6, 50; regla 7, 51; regla 8, 51.
\textbf{MODIFICADORES DE TONO}, lista, 141; barniz, 142; doblado de la tapa armónica, 143; doblado del fondo, 144; grosor de la tapa armónica, 144; arqueado, 147; arco alto, ejemplo, 148; leyes que rigen las líneas de recorrido de la onda sonora, 151; problema no resuelto en el arqueado, 152, la barra, 152; lobo causado por la mala posición de la barra, 153; lobo causado por la graduación, 154; posición de la barra que disminuye la potencia del D, 158; el poste, 160; el puente, 164; el diapasón, 165; las cuerdas, 181; bloque de refuerzo, 175; el violín tembloroso, 181; superficies interiores, 185; las salidas, 187; profundidad de las costillas, 199; la sordina, 302; crin del arco, 202.
\textbf{UNIFORMIDAD} de los valores de tono del violín, 131.
\textbf{VEREDICTO}, Etiqueta, Barniz y Precio vs Tono Dulce, 305.
\textbf{CARACTERÍSTICAS DEL TONO DEL VIOLÍN}, lugar de utilidad, 19.
\textbf{TRABAJOS DE CHAPADO}, 53.
\textbf{FENÓMENO DEL BARNIZ} No. I, 104.
\textbf{FENÓMENO DEL BARNIZ} No. II, 113.
\textbf{VIBRACIÓN}, normal y transversal, 291; velocidad comparada, 299.
\textbf{SEGMENTOS VENTRALES}, 295.
\textbf{LOBO}, causado por la barra, ejemplo, 153; causado por la graduación de la tapa armónica, ejemplo, 154.
\end{multicols}

\section*{Fe de Erratas}

\begin{itemize}
    \item Página 19, línea 8, en lugar de "tenora," lea tenoro.
    \item Página 44, línea 2 verso, en lugar de "Music," lea Music's.
    \item Página 48, línea 12, en lugar de "give," lea gives.
    \item Página 57, línea 24, en lugar de "govern's," lea governs.
    \item Página 96, línea 14, en lugar de "a basso," lea a bassa.
    \item Página 173, línea 1, en lugar de "diminish," lea increase.
    \item Página 216, línea 20, en lugar de "purfing," lea purfling.
    \item Página 217, líneas 1, 13, en lugar de "purfing," lea purfling.
    \item Página 297, línea 3, en lugar de "MOVEMENT," lea MOVEMENTS.
\end{itemize}