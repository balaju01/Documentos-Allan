\documentclass[a4paper,12pt]{article}
\usepackage[utf8]{inputenc}
\usepackage{geometry}
\usepackage{graphicx}
\geometry{top=2cm,bottom=2cm,left=2cm,right=2cm}

\begin{document}

% Membrete de la Asociación
\begin{flushleft}
    \includegraphics[width=0.15\textwidth]{logo.png} % Reemplaza 'logo.png' con el nombre del archivo de tu logo
    \hfill
    \begin{minipage}[t]{0.8\textwidth}
        \textbf{\large ASOCIACIÓN CIVIL MÚSICA Y BAILE TRADICIONAL A.C.} \\
        Calle Malva 166, Ampliación del Porvenir, Morelia, Michoacán \\
        Teléfono: [Teléfono de contacto] \\
        Correo Electrónico: [Correo de la Asociación]
    \end{minipage}
\end{flushleft}

\vspace{1cm}

\begin{center}
    \textbf{\Large CONVENIO DE COLABORACIÓN} \\[0.5cm]
    \textbf{\large ENTRE LA ASOCIACIÓN CIVIL MÚSICA Y BAILE TRADICIONAL A.C. Y ELVIRA ÁLVAREZ ISAIS} \\[0.5cm]
    \textbf{Morelia, Michoacán. [Fecha]}
\end{center}

\vspace{1cm}

En la Ciudad de Morelia, Michoacán, a [día] de [mes] de [año], se celebra el presente Convenio de Colaboración entre las siguientes partes:

\noindent
\textbf{I. MÚSICA Y BAILE TRADICIONAL,} asociación civil sin fines de lucro, con domicilio en Calle Malva 166, Ampliación del Porvenir, Morelia, Michoacán, representada en este acto por el Sr. David Durán Naquid, en su carácter de presidente, en adelante denominada \textbf{\"LA ASOCIACIÓN\"}.

\noindent
\textbf{II. La Sra. Elvira Álvarez Isais,} con domicilio en C. Francisco J. Mújica 796, Col. 22 de Octubre, 60650 Apatzingán, Michoacán, en adelante denominada \textbf{\"LA COLABORADORA\"}.

\vspace{0.5cm}

Ambas partes se reconocen mutuamente la capacidad legal para celebrar el presente convenio y acuerdan lo siguiente:

\vspace{0.5cm}

\hrule

\vspace{0.5cm}

\section*{CLÁUSULAS}

\subsection*{PRIMERA. OBJETO DEL CONVENIO}

El objeto del presente convenio es regular la disposición y uso de los recursos económicos donados por \textbf{LA COLABORADORA} a \textbf{LA ASOCIACIÓN}, con el propósito de garantizar que dichos recursos se utilicen de manera adecuada en actividades vinculadas a fines asistenciales y de promoción cultural.

\subsection*{SEGUNDA. DESTINO DE LOS RECURSOS}

1. \textbf{Distribución de los recursos:}
    \begin{itemize}
        \item El \textbf{40\%} de los recursos donados por \textbf{LA COLABORADORA} se destinarán exclusivamente a los programas o actividades que ella designe, siempre y cuando dichos programas estén alineados con los objetivos asistenciales y culturales de \textbf{LA ASOCIACIÓN}.
        \item El \textbf{60\%} de los recursos serán administrados y ejecutados directamente por \textbf{LA ASOCIACIÓN} para los proyectos que ya tiene comprometidos, en conformidad con su objeto social.
    \end{itemize}

2. \textbf{Justificación del uso de recursos:}
    \begin{itemize}
        \item Si \textbf{LA COLABORADORA} decide ejecutar directamente los programas que designe, deberá comprobar los gastos mediante la presentación de facturas u otros documentos fiscales válidos.
        \item En caso de que los programas designados sean operados por \textbf{LA ASOCIACIÓN}, \textbf{LA COLABORADORA} queda exenta de esta obligación de comprobación.
    \end{itemize}

\subsection*{TERCERA. COMPROMISOS DE LAS PARTES}

1. \textbf{Compromisos de LA ASOCIACIÓN:}
    \begin{itemize}
        \item Administrar y ejercer los recursos asignados al 60\% de manera transparente y conforme a su objeto social.
        \item Compartir con \textbf{LA COLABORADORA} un informe detallado de los resultados obtenidos con la ejecución de dichos recursos.
        \item Garantizar que los programas designados por \textbf{LA COLABORADORA}, en el caso de ser operados por \textbf{LA ASOCIACIÓN}, se ejecuten en tiempo y forma.
    \end{itemize}

2. \textbf{Compromisos de LA COLABORADORA:}
    \begin{itemize}
        \item Realizar los donativos en los tiempos y formas acordados entre ambas partes.
        \item Designar los programas o actividades a los que desea destinar el 40\% de los recursos, especificando su finalidad y alcance.
        \item Proporcionar comprobantes fiscales en caso de ejecutar directamente los programas seleccionados.
    \end{itemize}

\subsection*{CUARTA. VIGENCIA}

El presente convenio entrará en vigor a partir de la firma de ambas partes y tendrá una duración indefinida, hasta que cualquiera de las partes manifieste su deseo de darlo por terminado mediante notificación por escrito con al menos 30 días de anticipación.

\subsection*{QUINTA. MODIFICACIONES AL CONVENIO}

Cualquier modificación al presente convenio deberá realizarse por escrito y contar con la aprobación de ambas partes.

\subsection*{SEXTA. RESOLUCIÓN DE CONTROVERSIAS}

Las partes acuerdan que cualquier controversia derivada de la interpretación o cumplimiento del presente convenio será resuelta de manera amistosa. En caso de no llegar a un acuerdo, se someterá a la jurisdicción de los tribunales competentes de Morelia, Michoacán.

\vspace{0.5cm}
\hrule
\vspace{0.5cm}

En testimonio de conformidad, las partes firman el presente convenio en dos ejemplares de igual valor y efecto, quedando un ejemplar en poder de cada una de las partes.

\vspace{1cm}

\textbf{POR MÚSICA Y BAILE TRADICIONAL A.C.} \\
\rule{10cm}{0.4pt} \\
David Durán Naquid \\
Representante Legal

\vspace{1cm}

\textbf{POR LA COLABORADORA} \\
\rule{10cm}{0.4pt} \\
Elvira Álvarez Isais

\vspace{1cm}

\textbf{TESTIGOS:} \\[0.5cm]
\rule{10cm}{0.4pt} \\
Nombre: [Testigo 1] \\[1cm]
\rule{10cm}{0.4pt} \\
Nombre: [Testigo 2]

\end{document}
