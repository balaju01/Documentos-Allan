\documentclass[a4paper,10pt]{article}
\usepackage[utf8]{inputenc}
\usepackage{graphicx}
\usepackage{listings}
\usepackage{amsmath}
\usepackage{amssymb}

%opening
\title{Secreto Compartido de shamir}
\author{Allan y Jhon}

\begin{document}

\maketitle

\section{Secreto Compartido}
El secreto compartido, es un método dise\~nado para compartir un objeto entre un grupo de participantes. Este método fue propuesto por Adi Shamir en 1997.
\\
\\
El objetivo de este método es dividir un secreto $K$ en $w$ partes,que son dadas a $w$ participantes. Para recuperar el secreto es necesario tener al menos $u$
elementos de las $w$ partes siendo $u \leq w$. Y no es posible recuperar el secreto si se tienen menos que $u$ partes.
\\
\\
Para construir el esquema del secreto compartido primero es necesario seleccionar un $p \geq w+1$ el cual define el anillo $Z_p$.
\\
\\
El procedimiento para dividir un secreto $K$ en $w$ partes es el siguiente:

\begin{enumerate}
 \item Se seleccionan $w$ elementos distintos de cero del anillo $Z_p$ denotados como $x_i$ donde $1 \leq i \leq w$.
 \item Se seleccionan $u-1$ elementos aleatorios de $Z_p$ denotados como $a_1,...,a_{u-1}$.
 \item Se construye el polinomio $y_x$ de la siguiente forma. Sea \begin{equation}
            y(x)=K+\sum_{j=1}^{u-1} a_j x^j mod \quad p
           \end{equation}
           Por medio de este polinomio se calculan los elementos $y_i$.
 \item La salida es el conjunto $S=\{(x_1,y_1),...,(x_w,y_w)\}$.
\end{enumerate}

Para recuperar el secreto solo tenemos que resolver un sistema de ecuaciones que es definido por el polinomio característico $a(x)=a_0+a_1x+...+a_{u-1}x^{u-1}$.
\\
\\
Posteriormente se seleccionan $u$ pares de elementos $(x_w,y_w)$ con los que obtendremos nuestro sistema de ecuaciones a resolver. El elemento que nos interesa obtener del sistema de ecuaciones es $a_0$ ya que este es el valor de nuestro secreto $K$.
\\
\textbf{Ejemplo}
Se selccion\'o un anillo  $Z_p=11$\hspace{0,4cm} con \hspace{0,4cm}$w=5$\hspace{0,4cm} incógnitas de las que se resuelven\hspace{0,4cm}$u=2$.\hspace{0,4cm}Se seleccion\'o como llave\hspace{0,4cm}$k=8$\hspace{1cm}
\\ 
\\
Se selecciona los $u-1$ elementos del anillo $Z_p$\\
$a_1=5$ 
\\
\\
Del anillo $Z_p$ se seleccionan los $w$ elementos $x_i$\\
$x_1=2$\hspace{1cm}$x_2=7$\hspace{1cm}$x_3=9$\hspace{1cm}$x_4=10$\hspace{1cm}$x_5=3$\\
\\
Se calcula el conjunto de elementos $y_i$ por medio de la ecuaci\'on 
\begin{equation}
 y_i=k+\sum_{j=1}^{u-1} a_j x_i^j mod \quad p
\end{equation}
\\
Sustituyendo la sumatoria para nuestro caso queda
\\
\begin{equation}
 y_i=K+a_1x_i^1mod \quad p
\end{equation}
\\
$y_1=8+5(2)mod11=7$\hspace{3,5cm}$y_2=8+5(7)mod11=10$\hspace{3,5cm}$y_3=8+5(9)mod11=9$\hspace{3,5cm}$y_4=8+5(10)mod11=3$\hspace{3,5cm}$y_5=8+5(3)mod11=1$\\
\\
Se tienen los pares $S=(A_n(x_n,y_n))$\\
$A_1(2,7)$\hspace{1cm}$A_2(7,10)$\hspace{1cm}$A_3(9,9)$\hspace{1cm}$A_4(10,3)$\hspace{1cm}$A_5(3,1)$\\
\\
Para recuperar la llave $K$ es necesario seleccionar $u$ pares del conjunto $S$, los seleccionados son:\\
$A_2(7,10)$\hspace{1cm}$A_4(10,3)$\\
\\
Con estos pares podemos calcular un sistema de ecuaciones resolviendo el polinomio característico para $u=2$\\
$a_0+a_1x=y$ donde $a_0=k$\\
De lo que resulta el siguiente sistema de ecuaciones al sustituir los pares $A_2$ y $A_4$ en el polinomio\\
$a_0+7a_1=10$\\
$a_0+10a_1=3$\\
\\
Para resolver este polinomio podemos utilizar cualquiera de los métodos comunes que se usan en álgebra, solo que respetando el anillo $Z_p$, en este caso se resolverá por el método suma y resta.\\
\\
\begin{equation}
 a_0+7a_1=10
\end{equation}
\begin{equation}
 a_0+10a_1=3
\end{equation}
Multiplicamos la ecuación (3) por $-1$ y obtenemos el siguiente sistema:
\begin{equation}
 -a_0-7a_1=-10
\end{equation}
\begin{equation}
 a_0+10a_1=3
\end{equation}
sumamos la ecuación (5) + (6) dandonos como resultado:
\begin{equation}
 3a_1=4
\end{equation}
De la ecuación (7) despejamos $a_1$
\begin{equation}
 a_1=\frac{4}{3}
\end{equation}
Siendo 4 el inverso multiplicativo de 3 la ecuación (8) queda de la siguiente forma
\begin{equation}
 a_1=(4)(4)=16 mod11 =5
\end{equation}
Sustituimos $a_1$ en la ecuación (4) 
\begin{equation}
 a_0+10(5)=3
\end{equation}
Simplificamos y despejamos $a_0$
\begin{equation}
 a_0+(50mod11)=3
\end{equation}
\begin{equation}
 a_0+6=3
\end{equation}
\begin{equation}
 a_0=3-6
\end{equation}
\begin{equation}
 a_0=-3mod11=8
\end{equation}

Como $a_0=8$ podemos ver que esto es verdad por que $a_0=K$ y el $K$ que seleccionamos es $K=8$ con lo que recuperamos $K$ exitosamente.

\end{document}
