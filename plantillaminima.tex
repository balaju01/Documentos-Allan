\documentclass[xcolor=pdftex, x11names,table]{book}
 % Paquete de estilo 
\usepackage{RevistaMatematica_ITCR_Estilo_Libro_A} 
% Dimensiones ancho =18 cm, largo 22 cm
\usepackage[centering,text={18cm,22cm},showframe=false]{geometry}


%-----------------------------------------------------------------------
% COMANDOS PERSONALES
%-----------------------------------------------------------------------
\newcommand{\R}{\mathbb{R}}
\newcommand{\Z}{\mathbb{Z}}
\newcommand{\Q}{\mathbb{Q}}
\newcommand{\N}{\mathbb{N}}
\newcommand{\I}{\mathbb{I}}
\newcommand{\raya}{\rule{2cm}{0.01cm}\\}
\newcommand{\ds}{\displaystyle}
\newcommand{\sen}{\mathop{\rm sen}\nolimits}
\newcommand{\senh}{\mathop{\rm senh}\nolimits}
\newcommand{\arcsen}{\mathop{\rm arcsen}\nolimits}
\newcommand{\arcsec}{\mathop{\rm arcsec}\nolimits}
\newcommand{\bc}{\begin{center}}
\newcommand{\ec}{\end{center}}
\newcommand{\be}{\begin{enumerate}}
\newcommand{\ee}{\end{enumerate}}
% Aqu� podr�a cambiar el color de los entornos
\definecolor{colorejemplo}{RGB}{77,190,208}   % celestepastel
\definecolor{colordefinicion}{RGB}{104,48,39} % caf�
\definecolor{colorteorema}{RGB}{201,148,199}  % rosado pastel
\definecolor{colorcaja}{RGB}{160,128,104}     % caf� claro
\definecolor{colortitulo}{RGB}{0,133,202}
% --

%-----------------------------------------------------------------------
% Autor y T�tulo
%-----------------------------------------------------------------------
\usepackage{palatino}  % Fuente ppl
% TITULO DEL LIBRO
\title{Plantilla "A" para la edici�n de libros con LaTeX.\\
       \fntb[ppl][14]{Versi�n 1.0 -- Octubre 4, 2013.}
       }

% AUTOR (\fntx es un tipo de fuente definida en el paquete de estilo)
\author{
        \fntb[ppl][18]{Walter Mora F.}\\
        \fntg[phv][12]{wmora2@itcr.ac.cr }\\
        \fntg[phv][12]{Escuela de Matem�tica}\\
        \fntg[phv][12]{Instituto Tecnol�gico de Costa Rica}  \\      
      }
%----------------------------------------------------------------------

\usepackage{fourier} % usando fuentes "Fourier" en vez de "Palatino" 



\begin{document}
%-----------------------------------------------------------------------
% Hacer T�tulo
%-----------------------------------------------------------------------
\pagenumbering{roman}
\maketitle


%-----------------------------------------------------------------------
% Tabla de contenidos
%-----------------------------------------------------------------------
\tableofcontents

%-----------------------------------------------------------------------
% Inicio
%-----------------------------------------------------------------------
\pagenumbering{arabic}                 % Numeraci�n arabiga
\ansj=1                                % Cap 1 inicializa listas 
%--
%-----------------------------------------------------------------------
% Cap�tulo 1
%-----------------------------------------------------------------------
\chapter{Ejemplos con plantilla "A"}

\begin{wdefinicion}[(Divisibilidad)]
Sean $a,b$ enteros con $b \not = 0.$\\
\be
\item Decimos  que $b$ divide a $a$ si existe un entero $c$ tal que $a=bc.$\\
\item Si $b$ divide a $a$ escribimos $b|a$
\ee
\end{wdefinicion}


\begin{wejemplo}
   Sean $a,b,d \in \Z.$  Muestre que si $a|d$ y $d|b$ entonces $a|b$\\

   {\bf Soluci�n:} Si $a|d\;\wedge\; d|b\; \Longrightarrow\; 
                  d=k_1a \; \wedge \;      b=k_2d,  \;\;\mbox{con}\;\; k_1,k_2 \in \Z.$\\

   Luego  $b=k_2d=k_2(k_1a)\;\Longrightarrow\; a|b$
\end{wejemplo}

\begin{wteorema}[(Divisibildad)]
 Sean $a,b,d,p,q \in \Z.$
 \be
 \item Si $d|a$ y $d|b$ entonces $d|(ax+by)$ para cualquier $x,y \in \Z$

 \item Si $d|(p+q)$ y $d|p \;\; \Longrightarrow \;\;d|q.$

 \item Si $a,b \in \Z^+$ y $b|a\;\Longrightarrow\;a \geq b$

 \item Si $a|b,$ entonces $a|mb,$ con $m \in \Z.$

 \item Si $a,b \in \Z,$   $a|b$ y $b|a\;\Longrightarrow\;|a|=|b|$
 \ee
\end{wteorema}

\begin{wcorolario}[corolario1]
 Sea $n\in \Z,\;n>1.$ El m�s peque�o divisor positivo $d > 1$ de $n$ es primo.
\end{wcorolario}

\begin{wlema}[--- (El divisor m�s peque�o).][lema1]
 Sea $n\in \Z,\;n>1.$ El m�s peque�o divisor positivo $d > 1$ de $n$ es primo.
\end{wlema}


\begin{nota}
$0^0$ no est� definido, aunque a veces se conviene en que $0^0=1$, como en $\ds e^x=\sum_{n=0}^{\infty}\frac{x^n}{n!}$.
\end{nota}

El entorno para el vocabulario es simple (aunque en el c�digo del archivo de dise�o esta preparado para tener caja).\\



\begin{vocabulario}[(Funci�n suave).] 
Se dice que una funci�n ...
\end{vocabulario}

\begin{ejercicio}
Resolver $|\cos(\theta)|=1$  con $\theta \in\, \R.$ 
\end{ejercicio}




\chapter{Tablas}
Es el entorno usual,



\begin{center}
\rowcolors{1}{}{gray!20}
\begin{tabular}{lcl}
\rowcolor{LightBlue2}$x_i$ & & $y_i=f(x_i)$\\   \hline
$x_0=0$                    & & $0$\\
$x_1=0.75$                 & & $-0.0409838$\\
$x_2=1.5$                  & & $1.31799$\\     \hline
\end{tabular}
\end{center}

\medskip
Note que se us� el color {\tt  LightBlue2} del modelo {\tt  x11names} del paquete {\tt  xcolor}\\

\section{Tablas con el paquete TIKZ}

En el archivo de estilo est� definido el entorno {\tt dataTable} para generar tablas usando Tikz (idea original de O. Lemaire, \url{http://olivierlemaire.wordpress.com/2010/03/08/tableaux-tikz/?})


\begin{center}
\begin{dataTable}{cll}%
{\white $i$}  & {\white $x_i$} & {\white $y_i=f(x_i)$} \\ \midrule[0pt]
1 & $x_0=0$                    & $0$\\            \midrule
2 & $x_1=0.75$                 &  $-0.0409838$\\  \midrule
3 & $x_2=1.5$                  &  $1.31799$\\     
\end{dataTable}
\captionof{table}{Tabla usando Tikz}
\end{center}

El c�digo es

\begin{lstlisting}
% El entorno est� definido en el archivo de estilo.
\begin{center}
\begin{dataTable}{cll}%
{\white $i$}  & {\white $x_i$} & {\white $y_i=f(x_i)$} \\ \midrule[0pt]
1 & $x_0=0$                    & $0$\\            \midrule
2 & $x_1=0.75$                 &  $-0.0409838$\\  \midrule
3 & $x_2=1.5$                  &  $1.31799$\\     
\end{dataTable}
\captionof{table}{Tabla usando Tikz}
\end{center}
\end{lstlisting}


 
 \clearpage
\thispagestyle{empty}
\addcontentsline{toc}{section}{\color{azulF} Bibliograf�a}
 \begin{thebibliography}{AAAAAA}% define el tamano de la columna izquierda
 \bibitem{Gautschi} W. Gautschi. {\em Numerical Analysis. An Introduction.}
 		Birkh\"{a}user, 1997.
 \bibitem{Henrici} P. Henrici.{\it Essentials of Numerical Analysis.}
 		 Wiley, New York, 1982.
  \end{thebibliography}
  
 

\end{document}


