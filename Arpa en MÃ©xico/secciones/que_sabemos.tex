\section{Que Sabemos}
Por falta de evidencia en la existencia de instrumentos de cuerda en las culturas precolombinas (excepto por el arco musical cuya procedencia aun esta en discusión) podemos afirmar que el arpa es una introducción echa por los conquistadores españoles a su llegada al nuevo mundo. La presencia de este instrumento se hizo patente desde los primeros días de la colonia es evidencia de esto la presencia  de mais Juan "el de el arpa" el cual es mencionado por Bernal Días del Castillo y es el primer arpero del que se tiene conocimiento.

\subsection{Tipos de Arpa}

Existe una polémica sobre el numero de arpas que llegaron a la Nueva España, algunos autores afirman que fueron dos y algunos otros que fueron tres. Donde se coincide es que llegaron durante los siglos XVI y XVII principalmente dos tipos de arpa primero un arpa de pequeñas dimensiones con apenas 24 cuerdas caja de una sola pieza y las patas que la soportan parte de la misma caja, de afinación  diatónica esta arpa se acostumbraba en la España del medievo en un conjunto conformado por arpa rabel y vihuela, posteriormente un arpa mas grande con un numero mayor de cuerdas, esta podía ser diatónica o cromática (de ordenes dobles cruzados), igualmente conservaba las patas como una prolongación del cuerpo del arpa.  Desafortunadamente en México no se ha conservado ninguna arpa de esa época por lo que toda la información que se conose de su organología y forma ha sido principalmente por la iconografía musical que se encuentra en las iglesias pero en mayor medida en el arte menor que retrata muchas estampas del México colonial y sus costumbres, también son referenciadas en crónicas escritas por viajeros, actas de cabildos y documentos inquisitoriales.

\subsection{Propagación y adopción cultural}

El instrumento en cuestión fue traído por soldados y clérigos, estos se encargaron de propagarlo por toda la Nueva España esto gracias a las misiones de evangelización en las cuales se instruía a los nativos mexicanos en su ejecución y también su construcción. Este proceso de evangelización creo una mixtura al rededor del instrumento, aun que el arpa fue traída con el fin de su uso en el culto católico los nativos mexicanos crearon un sincretismo alrededor de esta lográndose así una combinación de signos, significados, repertorios e incluso una reinterpretación y adaptación de su constitución.