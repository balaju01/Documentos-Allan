\section{Introducción}
Siendo el arpa uno de los primeros instrumentos musicales del que se tiene registro, pudiendo rastrear su influencia en culturas antiguas como la Egipcia o Hebrea, es evidente que con el paso del tiempo este ha sufrido modificaciones dejando su huella en comunidades, técnicas y repertorios. A su llegada al nuevo mundo, la proliferación que tuvo en el territorio Mexicano gracias a las misiones de evangelización obligo a que se adecuara a las necesidades musicales que surgían en ese momento en cada región, aumentando su tamaño, cambiando su sonido, dando y recibiendo prestamos en su constitución física, haciendo variaciones en sus materiales así como en las técnicas empleadas en su construcción, pasando de repertorio en repertorio. En la actualidad México goza de la mayor variedad de arpas populares en el mundo pero hay que tomar en cuenta que a pesar de esta rica extensión que ha tenido por el territorio hay excepciones donde este instrumento ha caído en el olvido he incluso desaparecido del gusto popular, por lo que surge la pregunta, Realmente ¿cuantas y cuáles son las arpas que existieron en México? y ¿Que repertorios y géneros populares en la antigüedad fueron tocados también con arpa?