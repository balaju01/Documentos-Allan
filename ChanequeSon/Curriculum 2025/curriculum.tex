\documentclass{article}
\usepackage[a4paper,margin=1in]{geometry}
\usepackage{hyperref}
\geometry{a4paper, margin=1in}
\usepackage{enumitem}

\begin{document}

\begin{center}
    \textbf{\Large ChanequeSon} \\
    \textbf{\large Música tradicional de la Tierra Caliente}
\end{center}

\textbf{Lugar de Procedencia:} San Pedro Ixtapan Copuyo, Tzitzio, Michoacán, México \\
\textbf{Año de Fundación:} 2014 \\
\textbf{Género Musical:} Música tradicional de la Tierra Caliente \\
\textbf{Integrantes:} Allan Ulises Zepeda Ibarra, David Durán Naquid, Elizabeth Avendaño Sayagua, Felix Omar Ruiz Sanchez, Enrique lopez villanueva, Cerissa Viridiana Herrera Oceguera.

\section*{Trayectoria Artística (2018-2024)}

\subsection*{2024}
\begin{itemize}[leftmargin=*]
    \item Participación en el Festival Cultural Tarandacuao (Guanajuato).
    \item Participación en el Encuentro Nacional de Mariachis Tradicionales (Guadalajara).
    \item Presentación en el 5to Festival Aires de la Costa Cuilala (Michoacán, noviembre).
    \item Participación en el Encuentro de Pueblos Negros (Temixco, Morelos, noviembre).
    \item Participación en el Encuentro Afrodescendiente "Voces y Raíces de Michoacán" (julio).
    \item Presentación en el Festival Música de Tierra Caliente en el Complejo Cultural Los Pinos (CDMX, julio).
\end{itemize}

\subsection*{2023}
\begin{itemize}[leftmargin=*]
    \item Participación en el Festival Son para Milo (CDMX).
    \item Presentación en la Verbena Navideña en el Zócalo (CDMX).
    \item Participación en el XIX Foro de Música Tradicional de la Fonoteca INAH (CDMX).
    \item Presentación en el 50 Aniversario de la Facultad de Historia de la Universidad Michoacana de San Nicolás de Hidalgo (Morelia, Michoacán, octubre).
    \item Clausura del Diplomado La Pixca, Cultura Comunitaria para la Paz (Morelia, Michoacán, septiembre).
    \item Participación en el Festival "Sones, Versos y Jarabes" (Tacámbaro, Michoacán, mayo).
\end{itemize}

\subsection*{2022}
\begin{itemize}[leftmargin=*]
    \item Presentación del disco "El Son redoblado. Memorias sonoras de un campesino".
    \item Participación en el Festival del Huapango (San Luis Potosí).
    \item Presentación en el Foro de la Música Tradicional del Museo Nacional de Antropología e Historia.
    \item Participación en el Festival Vida y Son (Tixtla, Guerrero).
    \item Colaboración en la obra teatral "A latidos del tambor" de Julio César García en el Centro Cultural Universitario de la UMSNH.
\end{itemize}

\subsection*{2021}
\begin{itemize}[leftmargin=*]
    \item Participación en el Festival del Tesechoacán (Veracruz).
    \item Presentación en el Encuentro Nacional del Mariachi Tradicional (Jalisco).
\end{itemize}

\subsection*{2020-2021}
\begin{itemize}[leftmargin=*]
    \item Adaptación a formatos virtuales durante la pandemia, realizando presentaciones en dúo, trío y cuarteto.
\end{itemize}

\subsection*{2019}
\begin{itemize}[leftmargin=*]
    \item Presentación de la propuesta escénica "¡Puro Tierra Caliente primo!" en La Casona del Teatro (Morelia).
\end{itemize}

\subsection*{2018}
\begin{itemize}[leftmargin=*]
    \item Participación en el aniversario de la radio comunitaria XETUMI (Tuxpan, Michoacán).
\end{itemize}

\section*{Formación y Becas}

\subsection*{2023}
\begin{itemize}[leftmargin=*]
    \item Beca Músicos Tradicionales Mexicanos otorgada por la Secretaría de Cultura Federal.
\end{itemize}


\section*{Grabaciones y Colaboraciones}

\subsection*{2024}
\begin{itemize}[leftmargin=*]
    \item Colaboración con el grupo Oro Negro (octubre a diciembre).
\end{itemize}

\subsection*{2023}
\begin{itemize}[leftmargin=*]
    \item Grabación del corrido "La Decapitación de Villa".
    \item Grabación de audio y video para diversas conferencias.
    \item Lanzamiento de la pieza "Cariño Sin Condición" (mayo).
    \item Lanzamiento del son "La Gallina" (julio).
    \item Lanzamiento de la pieza "El Toro que Hace Llover" (julio).
\end{itemize}

\subsection*{2022}
\begin{itemize}[leftmargin=*]
    \item Lanzamiento del disco "El Son Redoblado. Memorias Sonoras de un Campesino" (mayo).
    \item Participación en el proyecto "Voces y Raíces de la Ciudad" bajo el programa PACMYC, documentando procesos creativos de grupos de música tradicional en Michoacán.
\end{itemize}

\subsection*{2017}
\begin{itemize}[leftmargin=*]
    \item Colaboración en el disco "Ahí Les Va Otro Toro Puntall" con Elías Gamiño y su conjunto "Los Campiranos de Sur".
\end{itemize}

\subsection*{2013}
\begin{itemize}[leftmargin=*]
    \item Participación en la grabación del disco "Balcones de la Tierra Caliente", que muestra el repertorio tradicional de la región.
\end{itemize}

\section*{Prensa y Difusión}

\subsection*{2022}
\begin{itemize}[leftmargin=*]
    \item Entrevista en "Music Planet: Road Trip" de la BBC.
    \item Aparición en "Giros Michoacán" del Canal 13.
\end{itemize}

\subsection*{2021}
\begin{itemize}[leftmargin=*]
    \item Mención en "La Jornada" y "Vida en Xalapa".
\end{itemize}

\subsection*{2020}
\begin{itemize}[leftmargin=*]
    \item Participación en el programa "RedLab" de vinculación cultural.
\end{itemize}

\end{document}