\section{El arpa durante el siglo XIX}

Ya en pleno 1800 la presencia del arpa se hacia notar en todos los sectores de la sociedad mexicana, tanto en el ámbito religioso como popular, inclusive para el entretenimiento de las clases aristócratas de aquel entonces. \\ Como testimonio de esto podemos encontrar crónicas, historias y personajes en los cuales ya se puede vislumbrar la consolidación de varios generos que permanecen hasta la actualidad, e incluso hay testimonio de la aparición del arpa en generos en los que hoy en día ya no esta presente o estan en peligro de desaparecer.

\subsection{Ámbitos de uso}

La mención del arpa en diversos documentos de la época nos dejan vislumbrar su uso en los ámbitos populares y de alta sociedad, testimonio de esto es la carta enviada en 1843 a Guillermo Prieto de parte de Manuel Payno en donde describe una tertulia acontecida en Jalapa donde señoritas de la alta sociedad jalapeña con arpa y jaranas ejecutan valses de compositores alemanes alternandolos con sones como el jarabe descrito en la carta. \\ Otro testimonio de esto es dado por José Zorrilla publica en 1879 su ensayo México y los mexicanos en el cual comenta a su llegada a México el encuentro con diferentes orquestas típicas (nacientes en el siglo XIX) donde encuentra arpas que tienen pedales y arpas que no los tienen, haciendo una clara separación entre los que son usados en las calles y los que son usados en los teatros.\\El registro de interpretes y compositores nos da una idea de la gran popularidad que tenia el arpa y sus diferentes ámbitos de uso, ejemplo de esto son  José de Jesús Rosas Rosas arpista originario de Santa Cruz de Galeana (hoy Santa Cruz de Juventino Rosas) Guanajuato este no era otro que el padre de Juventino Rosas el señor José de Jesús Rosas Rosas junto a Juventino y sus demás hijos tenían una orquesta familiar que formo a inicios de la década de 1870, con la que tocaba en las fiestas populares de su pueblo y posteriormente viajaron a la Ciudad de México en 1875 a buscar mejores oportunidades tocando para la clases sociales altas. Otro ejemplo de esto es Genaro Codina Zacatecano que paso a la historia por ser el compositor de la Marcha de Zacatecas, que se decía tenia un virtuosismo especial con el arpa el cual era su instrumento predilecto. Juan Bartolo Tavira oriundo de Ajuchitlán Guerrero considerado el más famoso repentista que ha tenido la Tierra Caliente, era un experto en la ejecución del arpa (se afirma que fue el mejor), y el director de un conjunto integrado por violín, tamborita, guitarra panzona y arpa que se comenta tocaba en fandangos y fiestas patronales. Tomás León pianista mexicano compuso en 1872 el Jarabe Nacional siendo esta la primer obra para arpa de pedales compuesta en México.\\
En 1884 Carlos Curti funda la Orquesta Típica Mexicana que estaba integrada por un arpa, dos violines, una viola, dos violoncellos, una flauta, siete bandolones, dos salterios y dos bajos de cuerda.

\subsection{Las arpas del XIX}

Existen algunos testimonios visuales donde podemos apreciar apariencia del arpa y sus rasgos. Una de estas es un grabado de José Guadalupe Posada en donde se aprecia a La Orquesta Típica Zacatecana de Señoritas, en el se puede observar una arpa en la parte superior que si bien no se aprecia si es de pedales o no el diapasón de la misma sugiere que es cromática.\\
El testimonio no es tan amplio ya que durante este siglo apenas va naciendo este medio visual, pero podemos encontrar una foto de la Orquesta Típica Mexicana "Lerdo de Tejada", en donde se puede apreciar un arpa cromática.

\subsection{La transicion entre los siglos}
Si bien la mayoría de testimonio de la existencia del arpa están situadas de mediados hacia finales del siglo XIX, podemos suponer que a inicios del mismo las costumbres musicales no eran muy diferentes que a finales del siglo XVIII por lo que  muy seguramente el uso del arpa seguía vigente tanto en el uso litúrgico como en el popular. Esto sucede de igual manera al llegar el siglo XX ya que podemos encontrar muchas fotos de orquestas típicas y fiestas populares de antes durante y despues de la Revolución Mexicana.