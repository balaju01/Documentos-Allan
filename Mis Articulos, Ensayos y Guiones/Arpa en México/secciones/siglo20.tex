\section{siglo XX}
El Siglo XX como gestor de cambios trajo una gran revolucion a las costumbres musicales y encaminando al arpa a los usos que hoy en dia conocemos. Con los avances tecnologicos, los nuevos medios de entretenimiento y las musicas nuevas que fueron surgiendo generaron un parte aguas en los usos musicales de México, esto es de especial notoriedad, despues de la decada de los 50.

\subsection{Porfiriato, Revolucion y Primeros años del México moderno}
Las Orquestas Tipicas fueron un nicho importante para la promocion del arpa de manera nacional e internacional, como un medio de promocion de un nacionalismo oficial, creando asi los primeros estereo tipos del mexicano. En las Tipicas mas famosas siempre se encontraba un arpa cromatica de pedales, existen algunos registros fotograficos que nos dejan apreciar al conjunto y sus instrumentos.
\\
Por parte de la musica popular no hay tantos registros de el uso del arpa a principios, pero es comun el dato de el arpa de la Revolucion mexicana usada por corridistas para llevar las noticias de un lado a otro. La descripcion que es comun encontrar es la de un arpa de pequeñas dimenciones, facil de transportar y que varios apuntan que tipicamente esta arpa era usada en el bajio principalmente en Guanajuato. Aun que tambien hay datos de que arpas con estas caractristicas fueron usadas en el norte como lo deja ver las investigaciones que se han realizado sobre el origen del corrido del "Caballo Mojino" donde se menciona  a un arpero ciego llamado Tanilo fue el autor de la musica para dicho corrido. 