\documentclass[a4paper,10pt]{article}
\usepackage[utf8]{inputenc}
\usepackage[spanish]{babel}
\usepackage[left=2cm, right=2cm, top=2cm, bottom=2cm]{geometry}
\usepackage{enumitem}
\usepackage{xcolor}
\usepackage{titling}
\usepackage{titlesec}

\definecolor{darkblue}{rgb}{0.0, 0.0, 0.55}

\titleformat{\section}
  {\normalfont\Large\bfseries\color{darkblue}}{\thesection}{1em}{}
\titleformat{\subsection}
  {\normalfont\large\bfseries\color{darkblue}}{\thesubsection}{1em}{}
\titleformat{\subsubsection}
  {\normalfont\normalsize\bfseries\color{darkblue}}{\thesubsubsection}{1em}{}
  
\setlength{\droptitle}{-4em}
\pretitle{\begin{center}\Huge\bfseries\color{darkblue}}
\posttitle{\end{center}}
\preauthor{\begin{center}\large\bfseries\color{darkblue}}
\postauthor{\end{center}}
\predate{}
\postdate{}

\title{Currículum Vitae de Allan Ulises Zepeda Ibarra}
\date{}

\begin{document}

\maketitle

\section*{Información Personal}
Nombre: Allan Ulises Zepeda Ibarra \\
Fecha de Nacimiento: 31 de julio de 1992 \\
Lugar de Nacimiento: Ciudad de México

\section*{Experiencia Musical}
\begin{itemize}[left=0pt]
    \item Estudios de guitarra en CEDART Luis Spota Saavedra desde 2010, adquiriendo bases teóricas musicales.
    \item Participación en el taller de música folklórica de la Escuela Superior de Cómputo.
    \item Maestro de música en el proyecto comunitario Jóvenes Orquestas Orquestando la Lucha desde 2014 a 2019.
    \item Exhibición de trabajos musicales en diversos géneros y presentaciones en eventos nacionales e internacionales.
    \item Productor y arreglista en diversos proyectos de musica mexicana
    \item Laudero con experiencia en reparacion y construccion de intrumentos tradicionales y modernos
    \item Actualmente cursa la licenciatura en composicion, produccion y gestion en el colegio Andrew Bell
\end{itemize}

\section*{Estudios y Formación}
\begin{itemize}[left=0pt]
    \item \textbf{2012:} Diplomado en Taller de Ensambles Musicales, Ciudad de México (INBA).
    \item \textbf{2013:} Técnico Instrumentista, Ciudad de México (CEDART Luis Spota Saavedra).
          Taller teórico-práctico Construcción de Guitarra Popular.
          Ritmos y repertorio avanzado del Son Jarocho.
    \item \textbf{2014:} Taller de ensamble y ritmos Mexicanos y Latinoamericanos.
          Ritmos tradicionales Norestenses.
          Nomenclatura musical numérica.
          Construcción de jaranas con técnica hermanada.
    \item \textbf{2015:} Diversos talleres de ritmos y repertorios tradicionales mexicanos.
    \item \textbf{2016:} Música y Ritmos Purépechas.
    \item \textbf{2017-2020:} Varios talleres y clases magistrales adicionales sobre musica latinoamericana.
    \item \textbf{2023:} Licenciatura en Composición, Producción y Gestión (Colegio Andrew Bell, en línea).
\end{itemize}

\section*{Presentaciones Destacadas}
\begin{itemize}[left=0pt]
    \item Diversas presentaciones musicales en eventos como Festival Son Para Milo, Feria Internacional del Libro del Zócalo, Palacio de Bellas Artes, entre otros.
    \item Participación en el circuito de presentaciones Festifolk España en 2017.
    \item Giras musicales por Colombia en 2018 y por Europa en 2019.
\end{itemize}

\section*{Coloquios y Ponencias}
\begin{itemize}[left=0pt]
    \item Ponencias en 2021 y 2022 sobre la transcripción de cuadernos de Don Alberto Albarran y el estudio de la guitarra panzona desde el diseño sonoro.
\end{itemize}

\section*{Producciones Musicales}
\begin{itemize}[left=0pt]
    \item Producción y arreglo del disco \textit{Jovenes Orquestas: Lucierngas y Cocuyos} (2016).
    \item Producción, interpretación y arreglo del disco \textit{Sonalli} (2017).
    \item Producción e interpretación en la sesión desde Radio Educación para \textit{Sonalli} (2019).
    \item Grabación y producción del disco \textit{ChanequeSon: El son redoblado memorias sonoras de un campesino} (2021).
    \item Grabación, mezcla y producción de campo para \textit{Franco Torres Blancas} (2022).
\end{itemize}


\end{document}
